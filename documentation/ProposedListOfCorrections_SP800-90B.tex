\documentclass[a4paper,xelatex,english]{bxjsreport}
\usepackage{tikz}%(これで、pgf と pgffor も読み込まれます。)
\usetikzlibrary{arrows.meta,bending,calc,shapes,positioning,shapes.callouts}
\usepackage{ascmac}
\usepackage{fancybox}
\usepackage{amsmath,amssymb}
\usepackage{algorithm}
\usepackage{algpseudocode}
\usepackage{paralist}
\usepackage{cases}
\usepackage{url}
\usepackage[open,openlevel=4]{bookmark}
\newcommand\mib[1]{\boldsymbol{#1}}

\usepackage{xcolor}
\usepackage{fancyhdr}
\usepackage[explicit]{titlesec}
\usepackage{xspace}

\usepackage[many]{tcolorbox}
%%%%%%%%%%%%%%%%%%%%%%%%%%%%%%%%%%%%%%%%%%%%%%%%%%%%%%%%%%%%%%%%%%%%%%%%%%%%%%%%
%%%
%%%
%%%
%%%%%%%%%%%%%%%%%%%%%%%%%%%%%%%%%%%%%%%%%%%%%%%%%%%%%%%%%%%%%%%%%%%%%%%%%%%%%%%%
\definecolor{mydarkblue}{RGB}{0,163,243}
\definecolor{mylightblue}{RGB}{191,233,251}
\definecolor{mediumtealblue}{rgb}{0.0, 0.33, 0.71}
%%%%%%%%%%%%%%%%%%%%%%%%%%%%%%%%%%%%%%%%%%%%%%%%%%%%%%%%%%%%%%%%%%%%%%%%%%%%%%%%
%%%
%%%
%%%
%%%%%%%%%%%%%%%%%%%%%%%%%%%%%%%%%%%%%%%%%%%%%%%%%%%%%%%%%%%%%%%%%%%%%%%%%%%%%%%%
%\def\chpcolor{blue!45}
\def\chpcolor{mydarkblue}
%\def\chpcolortxt{blue!60}
\def\chpcolortxt{mydarkblue}
\def\sectionfont{\sffamily\LARGE}

%%%%%%%%%%%%%%%%%%%%%%%%%%%%%%%%%%%%%%%%%%%%%%%%%%%%%%%%%%%%%%%%%%%%%%%%%%%%%%%
%%%%%%
%%%%%% 4段目までセクション番号を出力
%%%%%%
%%%%%%%%%%%%%%%%%%%%%%%%%%%%%%%%%%%%%%%%%%%%%%%%%%%%%%%%%%%%%%%%%%%%%%%%%%%%%%%
\setcounter{secnumdepth}{4}

\setcounter{tocdepth}{4}

%%%%%%%%%%%%%%%%%%%%%%%%%%%%%%%%%%%%%%%%%%%%%%%%%%%%%%%%%%%%%%%%%%%%%%%%%%%%%%%
%%%%%%
%%%%%%
%%%%%%
%%%%%%%%%%%%%%%%%%%%%%%%%%%%%%%%%%%%%%%%%%%%%%%%%%%%%%%%%%%%%%%%%%%%%%%%%%%%%%%
\makeatletter
%Section:
\def\@sectionstrut{\vrule\@width\z@\@height12.5\p@}
\def\@makesectionhead#1{%
  {\par\vspace{20pt}%
   \parindent 0pt\raggedleft\sectionfont
   \colorbox{\chpcolor}{%
     \parbox[t]{90pt}{\color{white}\@sectionstrut\@depth4.5\p@\hfill
       \ifnum\c@secnumdepth>\z@\thesection\fi}%
   }%
   \begin{minipage}[t]{\dimexpr\textwidth-90pt-2\fboxsep\relax}
   \color{\chpcolortxt}\@sectionstrut\hspace{5pt}#1
   \end{minipage}\par
   \vspace{10pt}%
  }
}
\def\section{\@afterindentfalse\secdef\@section\@ssection}
\def\@section[#1]#2{%
  \ifnum\c@secnumdepth>\m@ne
    \refstepcounter{section}%
    \addcontentsline{toc}{section}{\protect\numberline{\thesection}#1}%
  \else
    \phantomsection
    \addcontentsline{toc}{section}{#1}%
  \fi
  \sectionmark{#1}%
  \if@twocolumn
    \@topnewpage[\@makesectionhead{#2}]%
  \else
    \@makesectionhead{#2}\@afterheading
  \fi
}
\def\@ssection#1{%
  \if@twocolumn
    \@topnewpage[\@makesectionhead{#1}]%
  \else
    \@makesectionhead{#1}\@afterheading
  \fi
}
\makeatother
%%%%%%%%%%%%%%%%%%%%%%%%%%%%%%%%%%%%%%%%%%%%%%%%%%%%%%%%%%%%%%%%%%%%%%%%%%%%%%%%
%%%
%%%
%%%
%%%%%%%%%%%%%%%%%%%%%%%%%%%%%%%%%%%%%%%%%%%%%%%%%%%%%%%%%%%%%%%%%%%%%%%%%%%%%%%%
\newsavebox\mybox
\newlength\secnumwd

%\definecolor{mydarkblue}{RGB}{0,163,243}
%\definecolor{mylightblue}{RGB}{191,233,251}

\titleformat{\section}
  {\normalfont\Large\sffamily\color{mydarkblue}}
  {}
  {-5em}
  {%
    \savebox\mybox{\normalfont\Large\sffamily\color{mydarkblue}\bfseries\thesection}%
    \settowidth\secnumwd{\usebox\mybox}%    
    \parbox[t]{\secnumwd}{{\bfseries\thesection}}\hspace{1em}%
    \parbox[t]{\dimexpr\textwidth+5em-\secnumwd-1em\relax}{#1}%
  }
  [\vskip-1.75ex\hskip-5em{\color{gray!60}\titlerule[2pt]}]
\titleformat{name=\section,numberless}
  {\normalfont\Large\sffamily\color{mydarkblue}}
  {}
  {-5em}
  {#1}
  [\vskip-1.75ex\hskip-5em{\color{gray!60}\titlerule[2pt]}]

\newcommand\FrameBoxL[1]{%
  \fcolorbox{mylightblue}{mydarkblue}{\makebox[3cm][l]{\textcolor{white}{\bfseries#1}}}%
}
\newcommand\FrameBoxR[1]{%
  \fcolorbox{mylightblue}{mydarkblue}{\makebox[3cm][r]{\textcolor{white}{\bfseries#1}}}%
}

\pagestyle{fancy}
\fancyheadoffset[EL]{\dimexpr1in+\evensidemargin+\hoffset\relax}
\fancyheadoffset[OR]{\dimexpr\paperwidth-\oddsidemargin-1in-\textwidth-\hoffset\relax}
\fancyhf{}
\renewcommand\headrulewidth{0pt}
\fancyhead[OR]{\FrameBoxL{\thepage}}
\fancyhead[EL]{\FrameBoxR{\thepage}}
%%%%%%%%%%%%%%%%%%%%%%%%%%%%%%%%%%%%%%%%%%%%%%%%%%%%%%%%%%%%%%%%%%%%%%%%%%%%%%%%
%%%
%%% argmax
%%%
%%%%%%%%%%%%%%%%%%%%%%%%%%%%%%%%%%%%%%%%%%%%%%%%%%%%%%%%%%%%%%%%%%%%%%%%%%%%%%%%
\newcommand{\argmax}{\mathop{\textrm{arg~max}}\limits}
%%%%%%%%%%%%%%%%%%%%%%%%%%%%%%%%%%%%%%%%%%%%%%%%%%%%%%%%%%%%%%%%%%%%%%%%%%%%%%%%
%%%
%%% vertical equal
%%%
%%%%%%%%%%%%%%%%%%%%%%%%%%%%%%%%%%%%%%%%%%%%%%%%%%%%%%%%%%%%%%%%%%%%%%%%%%%%%%%%
\newcommand\rotateequal[1]{%
  \ifmmode
    \underset{#1}{\rotatebox{90}{$=$}}%
  \fi
}
%%%%%%%%%%%%%%%%%%%%%%%%%%%%%%%%%%%%%%%%%%%%%%%%%%%%%%%%%%%%%%%%%%%%%%%%%%%%%%%%
%%%
%%%
%%%
%%%%%%%%%%%%%%%%%%%%%%%%%%%%%%%%%%%%%%%%%%%%%%%%%%%%%%%%%%%%%%%%%%%%%%%%%%%%%%%%
\algnewcommand{\algorithmicgoto}{\textbf{go to}}%
\algnewcommand{\Goto}{\algorithmicgoto\xspace}%
\algnewcommand{\Label}{\State\unskip}
%%%%%%%%%%%%%%%%%%%%%%%%%%%%%%%%%%%%%%%%%%%%%%%%%%%%%%%%%%%%%%%%%%%%%%%%%%%%%%%%
%%%
%%%
%%%
%%%%%%%%%%%%%%%%%%%%%%%%%%%%%%%%%%%%%%%%%%%%%%%%%%%%%%%%%%%%%%%%%%%%%%%%%%%%%%%%
\definecolor{myblue}{RGB}{0,163,243}

\newtcolorbox[auto counter,number within=section]{mytheorem}[1][]{
  enhanced jigsaw,colback=white,colframe=myblue,coltitle=myblue,
  fonttitle=\bfseries\sffamily,
  sharp corners,
  detach title,
  leftrule=22mm,
  underlay unbroken and first={\node[below,text=white,font=\sffamily\bfseries,align=center]
    at ([xshift=-11mm,yshift=-1mm]interior.north west) {THEOREM\\\thetcbcounter};},
  breakable,pad at break=1mm,
  #1,
  code={\ifdefempty{\tcbtitletext}{}{\tcbset{before upper={\tcbtitle\par\medskip}}}},
}

%%%\usepackage{titlesec}
%%%%%%\usepackage{lipsum}
%%%\usepackage{tikz}\usetikzlibrary{shapes.misc}
%%%\newcommand\titlebar{%
%%%\tikz[baseline,trim left=3.1cm,trim right=3cm] {
%%%    \fill [cyan!25] (2.5cm,-1ex) rectangle (\textwidth+3.1cm,2.5ex);
%%%    \node [
%%%        fill=cyan!60!white,
%%%        anchor= base east,
%%%        %rounded rectangle,
%%%        rectangle,
%%%        minimum height=3.5ex] at (3cm,0) {
%%%        \textbf{\thesection.}
%%%    };
%%%}%
%%%}
%%%\titleformat{\section}{\large}{\titlebar}{0.1cm}{}
%%%\renewcommand*{\thesection}{\arabic{section}}

%%%\usepackage{xcolor}
%%%
%%%\setcounter{section}{1} % just to emulate a chapter has started
%%%
%%%\renewcommand*{\othersectionlevelsformat}[1]{%
%%%  \makebox[0pt][r]{%
%%%    \fcolorbox{cyan!60!white}{cyan!60!white}{\color{white}\csname the#1\endcsname}%
%%%    \enskip
%%%  }
%%%}


\renewcommand{\figurename}{Figure }
\renewcommand{\tablename}{Table }
\renewcommand{\refname}{References}
\renewcommand{\today}{\the\year/\the\month/\the\day}
\renewcommand{\contentsname}{Contents}
\renewcommand{\figurename}{Figure }
\renewcommand{\tablename}{Table }
\renewcommand{\refname}{References}
\renewcommand{\appendixname}{Appendix }

%%%%%%%%%%%%%%%%%%%%%%%%%%%%%%%%%%%%%%%%%%%%%%%%%%%%%%%%%%%%%%%%%%%%%%%%%%%%%%%%
%%%
%%%
%%%
%%%%%%%%%%%%%%%%%%%%%%%%%%%%%%%%%%%%%%%%%%%%%%%%%%%%%%%%%%%%%%%%%%%%%%%%%%%%%%%%
\author{Gen'ya SAKURAI}
\title{Proposed list of corrections for NIST SP 800-90B 6.3 Estimators}
\date{\today}

\begin{document}
\maketitle

%\setcounter{chapter}{1} % just to emulate a chapter has started

%\centerline{THIS UPDATE: Nov. 24, 2022}

%%%%%%%%%%%%%%%%%%%%%%%%%%%%%%%%%%%%%%%%%%%%%%%%%%%%%%%%%%%%%%%%%%%
%%%%%
%%%%%
%%%%%
%%%%%%%%%%%%%%%%%%%%%%%%%%%%%%%%%%%%%%%%%%%%%%%%%%%%%%%%%%%%%%%%%%%
\chapter{Proposed list of corrections for NIST SP 800-90B 6.3 Estimators}
%%%%%%%%%%%%%%%%%%%%%%%%%%%%%%%%%%%%%%%%%%%%%%%%%%%%%%%%%%%%%%%%%%%
%%%%%
%%%%%
%%%%%
%%%%%%%%%%%%%%%%%%%%%%%%%%%%%%%%%%%%%%%%%%%%%%%%%%%%%%%%%%%%%%%%%%%
\section{Introduction}
This list of corrections for NIST SP 800-90B \cite{SP80090B} 6.3 Estimators has been drafted so that an entropy estimating tool for claiming conformance can be developed in a traceable manner.
%%%%%%%%%%%%%%%%%%%%%%%%%%%%%%%%%%%%%%%%%%%%%%%%%%%%%%%%%%%%%%%%%%%
%%%%%
%%%%%
%%%%%
%%%%%%%%%%%%%%%%%%%%%%%%%%%%%%%%%%%%%%%%%%%%%%%%%%%%%%%%%%%%%%%%%%%
\section{Corrections to 6.3.4 The Compression Estimate}

\begin{enumerate}
	\item Correction to step 4-b-ii

\begin{shadebox}
\begin{algorithmic}
%%%%%
%%%%%
%%%%%
\State If $dict[s_{i}]$ is zero, add that value to the dictionary, i.e., $dict[s_{i}'] =i$. Let $D_{i-d} = i$.
%%%%%
%%%%%
%%%%%
\end{algorithmic}
\end{shadebox}
should be replaced by the following:
%\\ 
\begin{shadebox}
\begin{algorithmic}
%%%%%
%%%%%
%%%%%
\State If $dict[s_{i} \textcolor{red}{'}]$ is zero, add that value to the dictionary, i.e., $dict[s_{i}'] =i$. Let $D_{i-d} = i$.
%%%%%
%%%%%
%%%%%
\end{algorithmic}
\end{shadebox}

	\item Correction to the expression of $G(z)$

The Eq.(\ref{eq:Gin634}) should be replaced by Eq.\ref{eq:newGin634}.
\begin{align}
G(z) = \tfrac{1}{\nu} 
\sum_{t = d + 1}^{L} 
\sum_{u = 1}^{t} 
\log_{2}(u) F(z, t, u)  \label{eq:Gin634}
\end{align}
\begin{align}
G(z) = \tfrac{1}{\nu} 
\sum_{t = d + 1}^{ \textcolor{red}{\lfloor L / b \rfloor}} 
\sum_{u = 1}^{t} 
\log_{2}(u) F(z, t, u)  \label{eq:newGin634}
\end{align}

This correction makes sense if the summation over $t$ starts from $d + 1$ then it should end at $\lfloor L / b \rfloor$.  Also the factor $\tfrac{1}{\nu}$ coincides with this argument as $\nu = \lfloor L / b \rfloor - d$.

The Eq.(\ref{eq:newGin634}) can be further rewritten to as Eq.(\ref{eq:rewGin634}), as $\log_{2}(1) = 0$.
\begin{align}
G(z) = \tfrac{1}{\nu} 
\sum_{t = d + 1}^{\lfloor L / b \rfloor} 
\sum_{u = \textcolor{red}{2}}^{t} 
\log_{2}(u) F(z, t, u)  \label{eq:rewGin634}
\end{align}

Note that the r.h.s. of Eq.(\ref{eq:rewGin634}) can be optimized further for algorithmic efficiency.
\end{enumerate}
%%%%%%%%%%%%%%%%%%%%%%%%%%%%%%%%%%%%%%%%%%%%%%%%%%%%%%%%%%%%%%%%%%%
%%%%%
%%%%%
%%%%%
%%%%%%%%%%%%%%%%%%%%%%%%%%%%%%%%%%%%%%%%%%%%%%%%%%%%%%%%%%%%%%%%%%%
\section{Corrections to 6.3.10 The LZ78Y Prediction Estimate}


\begin{enumerate}
\item Missing step

The variable $C$ is used in step 4 without its definition.  So the following new step should be introduced just before step 4, and steps 4 to 6 should be renumbered accordingly.
\\
\begin{shadebox}
\begin{algorithmic}
%%%%%
%%%%%
%%%%%
\State Let $C$ be the number of ones in $correct$.
%%%%%
%%%%%
%%%%%
\end{algorithmic}
\end{shadebox}
\vspace{5mm}

\item Column header in Example

Also the column header 
%``\textbf{Max} ${\mib{D}}[{\mib{prev}}]$ {\bf entry}''
\begin{shadebox}
\begin{algorithmic}
%%%%%
%%%%%
%%%%%
\State \textbf{Max} ${\mib{D}}[{\mib{prev}}]$ \textbf{entry}
%%%%%
%%%%%
%%%%%
\end{algorithmic}
\end{shadebox}
 in the table for \textit{Example} should be replaced by 
%``$\argmax_{y} {\mib{D}}[{\mib{prev}}][y]$''
\begin{shadebox}
\begin{algorithmic}
%%%%%
%%%%%
%%%%%
\State $\argmax_{y} {\mib{D}}[{\mib{prev}}][y]$
%%%%%
%%%%%
%%%%%
\end{algorithmic}
\end{shadebox}
.
\end{enumerate}


%%%%%%%%%%%%%%%%%%%%%%%%%%%%%%%%%%%%%%%%%%%%%%%%%%%%%%%%%%%%%%%%%%%%%%%%%%%%%%%
%%%%%
%%%%% 参考文献
%%%%%
%%%%%%%%%%%%%%%%%%%%%%%%%%%%%%%%%%%%%%%%%%%%%%%%%%%%%%%%%%%%%%%%%%%%%%%%%%%%%%%
\begin{thebibliography}{99}
% 1
\bibitem{SP80090B}
Meltem S\"{o}nmez Turan,
Elaine Barker,
John Kelsey,
Kerry A. McKay,
Mary L. Baish,
Mike Boyle
\textit{Recommendation for the Entropy Sources Used for Random Bit Generation},
NIST Special Publication 800-90B, Jan. 2018
% 2
\bibitem{MathHandbook}
Franck W. J. Oliver,
Daniel W. Lozier,
Ronald F. Boisvert,
Charles W. Clark,
\textit{NIST Handbook of Mathematical Functions},
National Institute of Standards and Technology, 2010

\end{thebibliography}

\end{document}
